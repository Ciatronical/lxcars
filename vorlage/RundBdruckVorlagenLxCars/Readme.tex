%
% Bemerkungen zum Vorlagensatz RB von
% Richardson & Büren GmbH
%
% Hier wurden einige Ideen aufgegriffen, die in folgendem Vortrag
% erwähnt wurden:
% http://www.kivitendo-support.de/vortraege/Lx-Office%20Anwendertreffen%20LaTeX-Druckvorlagen-Teil3-finale.pdf
%
% Am Ende dieser Datei befindet sich ein Abschnitt "Quickstart", der die
% Vorbereitungsschritte beschreibt, die bei diesem Druckvorlagensatz nötig
% sind.
%
%
% Aufbau:
%   Die documentclass und alle usepackage-Anweisungen sind in
%   'inheaders.tex' ausgelagert. Diese werden von allen Vorlagen via
%   \input eingebunden.
%
%   Desweiteren sind einige Einstellungen und eigene Befehle, die alle
%   Vorlagen verwenden, in 'insetting.tex' untergebracht. Auch diese
%   werden mit \input eingebunden.
%   Da in eingebundenen Dateien die kivitendo-Variablen nicht aufgelöst
%   werden könnnen, werden die hier verwendeten Variablen in jedem
%   Dokument vorher mit \newcommand neu definiert.
%
% Sprachen:
%   In 'insettings.tex' wird anhand des verwendeten Sprachkürzels die
%   Sprache unterschieden und eine entsprechende Übersetzungsdatei geladen,
%   die Textbausteine  bzw. -Schnipsel enthält. Die Vorlagen verwenden nur
%   diese Schnipsel. Im Moment werden die Vorlagenkürzel DE und EN in
%   Benutzung mit den entsprechenden Übersetzungsdateien 'deutsch.tex'
%   und 'english.tex' unterschieden.
%
% Mandanten / Firma:
%   Um gleiche Vorlagen für verschiedene Firmen verwenden zu können, wird je
%   nach dem Wert der Kivitendo-Variablen <%kivicompany%> ein
%   Firmenverzeichnis ausgewählt (siehe 'insettings.tex'), in dem Briefkopf,
%   Identitäten und Währungs-/Kontoeinstellungen hinterlegt sind.
%   <%kivicompany%> enthält den Namen des verwendeten Mandantendaten.
%   Ist kein Firmenname eingetragen, so wird das
%   generische Unterverzeichnis 'firma' verwendet.
%
% Identitäten:
%    In jedem Firmen-Unterverzeichnis soll eine Datei 'ident.tex'
%    vorhanden sein, die mit \newcommand Werte für \telefon, \fax,
%    \firma, \strasse, \ort, \ustid, \email und \homepage definiert.
%
% Währungen / Konten:
%    Für jede Währung (siehe 'insettings.tex') soll eine Datei vorhanden
%    sein, die das Währungssymbol (\currency) und folgende Angaben für
%    ein Konto in dieser Währung enthält \kontonummer, \bank,
%    \bankleitzahl, \bic und \iban.
%    So kann in den Dokumenten je nach Währung ein anderes Konto
%    angegeben werden.
%    Nach demselben Schema können auch weitere, alternative Bankverbindungen
%    angelegt werden, die dann in insettings.tex als Variable im
%    unteren Abschnitt der Datei 'insettings.tex', Kommentar Fusszeile
%    (cfoot) eingefügt werden.
%
% Briefbogen/Logos:
%    Eine Hintergrundgrafik oder ein Logo kann in Abhängigkeit vom
%    Medium (z.B. nur beim Verschicken mit E-Mail) eingebunden
%    werden. Dies ist im Moment auskommentiert.
%
%    Desweiteren sind (auskommentierte) Beispiele enthalten für eine
%    Grafik als Briefkopf, nur ein Logo, oder ein komplettes DinA4-PDF
%    als Briefpapier.
%
% Fusszeile:
%    Die Tabelle im Fuß verwendet die Angaben aus firma/ident.tex und
%    firma/*_account.tex.
%
%
% Tabellen:
%    Als Tabellenumgebung wird longtable verwendet. Diese Umgebung
%    kann in einer Tabelle umbrechen. Da aber der Umbruch nicht von
%    kivitendo kontrolliert wird, kann man keinen Übertrag machen.
%    Innerhalb des Langtextes <%longdescription%> wird nicht umgebrochen. Um
%    dies zu erreichen kann z.B. per renewcommand das "\newline" umdefiniert
%    werden.

%
% Quickstart (wo kann was angepasst werden?):
%    insettings.tex : Pfad zu Angaben über Mandanten (default: firma)
%                     Logo/Briefpapier
%                     Seitenränder / Geometry
%                     Aussehen Kopf/Fußzeile
%    firma/*        : Angaben über Mandanten
% Es muß mindestens eine Sprache angelegt werden!
%    deutsch.tex    : Textschnipsel für Deutsch
%                     Dafür eine Sprache mit Vorlagenkürzel DE anlegen
%    english.tex    : Textschnipsel für Englisch
%                     Dafür eine Sprache mit Vorlagenkürzel EN anlegen
%
