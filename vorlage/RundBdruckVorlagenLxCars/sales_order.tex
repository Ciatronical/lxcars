\input{inheaders.tex}


% Variablen, die in settings verwendet werden
\newcommand{\lxlangcode} {<%template_meta.language.template_code%>}
\newcommand{\lxmedia} {<%media%>}
\newcommand{\lxcurrency} {<%currency%>}
\newcommand{\kivicompany} {<%employee_company%>}

% settings: Einstellungen, Logo, Briefpapier, Kopfzeile, Fusszeile
% Sprachüberprüfung
\ifthenelse{\equal{\lxlangcode}{EN}}{\input{english.tex}}{
  \ifthenelse{\equal{\lxlangcode}{DE}}{\input{deutsch.tex}}{\input{deutsch.tex}}
} % Ende EN


% Mandanten-/Firmenabhängigkeiten

% Pfad zu firmenspez. Angaben
% Hat man mehrere Mandanten muß man statt "Firma1" den Datenbanknamen seines
% Mandanten eingeben.

\IfSubStringInString{autoprofis}{\kivicompany}{\newcommand{\identpath}{autoprofis}}{
  \IfSubStringInString{autoprofis}{\kivicompany}{\newcommand{\identpath}{autoprofis}}
    {\newcommand{\identpath}{autoprofis}} % sonst
} % Ende Firma1

% Identität
\newcommand{\telefon} {033435-80 40 00}
\newcommand{\mobil} {0175-78 80 999}
\newcommand{\fax} {}
\newcommand{\firma} {Autoprofis}
\newcommand{\inhaber} {Geschäftsführer Ronny Zimmermann}
\newcommand{\strasse} {Bahnhofstr. 23}
\newcommand{\ort} {15345 Rehfelde}
\newcommand{\ustid} {DE230326189}
\newcommand{\stnr} {St.-Nr: 064/292/03463}
\newcommand{\finanzamt} {Finanzamt Strausberg}
\newcommand{\email} {Email: info@autoprofis24.de}
\newcommand{\emails} {info@autoprofis24.de}
\newcommand{\homepage} {Internet: www.autoprofis24.de}
\newcommand{\homepages} {www.autoprofis24.de}
\newcommand{\sonst} {Supported by LxCars}


% Währungen/Konten
\IfSubStringInString{USD}{\lxcurrency}{\input{\identpath/usd_account.tex}}{
  \IfSubStringInString{CHF}{\lxcurrency}{\input{\identpath/chf_account.tex}}{
    \IfSubStringInString{EUR}{\lxcurrency}{\newcommand{\currency}{\euro}
\newcommand{\kontoinhab}{Ronny Zimmermann}
\newcommand{\kontonummer}{35 0000 30 100}
\newcommand{\bank}{Sparkasse MOL}
\newcommand{\bankleitzahl}{170 540 40}
\newcommand{\iban}{DE081705404037}
\newcommand{\bic}{WEL}
}{\newcommand{\currency}{\euro}
\newcommand{\kontoinhab}{Ronny Zimmermann}
\newcommand{\kontonummer}{35 0000 30 100}
\newcommand{\bank}{Sparkasse MOL}
\newcommand{\bankleitzahl}{170 540 40}
\newcommand{\iban}{DE081705404037}
\newcommand{\bic}{WEL}
}
  } % Ende CHF
} % Ende USD

% Briefkopf, Logo oder Briefpapier
%%\IfSubStringInString{mail}{\lxmedia}{    % nur bei Mail
  % Nur ein Logo oben rechts
  \setlength{\wpYoffset}{130mm} % Verschiebung von der Mitte nach oben
  \setlength{\wpXoffset}{-51mm} % Verschiebung von der Mitte nach rechts
  \CenterWallPaper{0.38}{\identpath/AutoprofisLogo.png} % mit Skalierung
  % oder ganzer Briefbogen als Hintergrund
  %% \CenterWallPaper{1}{\identpath/briefbogen.pdf}
%%} % Mail-Ende


% keine Absätze nach rechts einrücken
\setlength\parindent{0pt}

% Papierformat, Ränder, usw.
\geometry{
        a4paper,      % DINA4
        %% left=19mm,    % Linker Rand
        width=182mm,  % Textbreite
        top=39mm,     % Abstand Textanfang von oben
        head=44mm,     % Höhe des Kopfes
        headsep=4mm, % Abstand Kopf zu Textanfang
        bottom=30mm,  % Abstand von unten
        %%showframe,    % Rahmen zum Debuggen anzeigen
}


% Befehl f. normale Schriftart und -größe
\newcommand{\ourfont}{\fontfamily{cmss}\fontsize{10pt}{12pt}\selectfont}


% Einstellungen f. Kopf und Fuss
\pagestyle{scrheadings}
\clearscrheadfoot
%\setheadwidth[20mm]{page} % Kopfzeile nach rechts verschieben
%\setfootwidth[-39mm]{page} % Fusszeile verschieben

% Befehl f. laufende Kopfzeile:
% 1. Text f. Kunden- oder Lieferantennummer (oder leer, wenn diese nicht ausgegeben werden soll)
% 2. Kunden- oder Lieferantennummer (oder leer)
% 3. Belegname {oder leer}
% 4. Belegnummer {oder leer}
% 5. Belegdatum {oder leer}
% Beispiel: \ourhead{\kundennummer}{<%customernumber%>}{\angebot}{<%quonumber%>}{<%quodate%>}
\newcommand{\ourhead}[5] {
\chead{
  \ifthenelse{\equal{\thepage}{1}}
    {}% then
    {\normalfont\fontfamily{cmss}\scriptsize
      \ifthenelse{\equal{#1}{}}{}{#1: #2 \hspace{0.7cm}}{}
      #3
      \ifthenelse{\equal{#4}{}}{}{~\nr: #4}
      \ifthenelse{\equal{#5}{}}{}{\vom ~ #5}
      \hspace{0.7cm} - \seite ~ \thepage/\pageref{LastPage} ~- }
}%ende chead
}

% Firmenfuss
\cfoot{
  {\normalfont\fontfamily{cmss} \tiny
     \begin{tabular}{p{5cm}p{4.5cm}lr}
        \firma                 & \email              & \textKontonummer & \kontonummer \\
        \strasse               & \homepage           & \textBank        & \bank \\
        \ort                   & \textUstid\ \ustid  & \textIban        & \iban \\
        \textTelefon~\telefon  & \finanzamt          & \textBic         & \bic \\
        Mobil~\mobil           & \sonst              & \textBankleitzahl& \bankleitzahl \\
     \end{tabular}
  }
}
%\cfoot{
%  {\normalfont\fontfamily{cmss} \tiny
%     \begin{tabular}{p{6cm}p{7cm}lr}
%        \firma                 & \textUstid\ \ustid  & \textKonto~\kontoinhab \\
%        \strasse               & \inhaber            & \textKontonummer~\kontonummer \\
%        \ort                   & \stnr               & \textBankleitzahl~\bankleitzahl \\
%        \textTelefon~\telefon  & \email              & \textBank~\bank \\
%        \textMobil~\mobil      & \homepage           & \textIban~\iban \\
%        \                      & \sonst              & \textBic~\bic \\
%     \end{tabular}
%  }
%}


% laufende Kopfzeile:
\ourhead{\kundennummer}{<%customernumber%>}{\auftragsbestaetigung}{<%ordnumber%>}{<%orddate%>}


\begin{document}

\ourfont
\begin{minipage}[t]{8cm}
  \scriptsize

  {\color{gray}\underline{\firma\ $\cdot$ \strasse\ $\cdot$ \ort}}
  \normalsize

  \vspace*{0.3cm}

  \textbf{<%name%>}

  <%cp_givenname%> <%cp_name%>

  <%street%>

  ~

  \textbf{<%country%>-<%zipcode%> <%city%>}

 
\end{minipage}

\begin{picture}(0,0)
  \put(344,155){           % Position 
    \begin{minipage}[t]{60mm}
      \small{
        Tel.:  \telefon \\
        Mobil: \mobil\\ 
        Web: \homepages\\
        Email: \emails
      }
    \end{minipage}
  }%Ende put
  \put(0,-578){\rule[-3mm]{180mm}{0.8pt}} % Linie über dem Footer
\end{picture}

\hfill
\begin{minipage}[t]{6.0cm}

  \vspace*{-2.9cm}

  \textbf{Auftragsnummer:}\hfill <%ordnumber%>

  \textbf{Auftragsdatum:}\hfill <%orddate%>

  \textbf{\kundennummer:}\hfill <%customernumber%>

  \textbf{\ansprechpartner:}\hfill <%employee_name%>

  <%if shipvia%> Stand des Wegstreckenzählers: \hfill  <%shipvia%><%end if%>

  <%if shippingpoint%> Amtliches Kennzeichen: \hfill  <%shippingpoint%><%end if%>

\end{minipage}

\vspace*{1.5cm}

\LARGE\textbf{Auftrag}

~

\normalsize 

%\hfill

% Bei Kontaktperson Anrede nach Geschlecht unterscheiden.
% Bei natürlichen Personen persönliche Anrede, sonst allgemeine Anrede.
\ifthenelse{\equal{<%cp_name%>}{}}{
  <%if natural_person%><%greeting%> <%name%>,<%else%>\anrede<%end if%>}{
  \ifthenelse{\equal{<%cp_gender%>}{f}}
    {\anredefrau}{\anredeherr} <%cp_title%> <%cp_name%>,}\\
\auftragsformel\\

\vspace{0.5cm}


\setlength\LTleft\parindent     % Tabelle beginnt am linken Textrand
\setlength\LTright{0pt}         % Tabelle endet am rechten Textrand
\begin{longtable}{@{}rrp{7cm}@{\extracolsep{\fill}}rrr@{}}
% Tabellenkopf
\hline
\textbf{\position} & \textbf{\artikelnummer} & \textbf{\bezeichnung} & \textbf{\menge} & \textbf{\einzelpreis} & \textbf{\gesamtpreis} \\
\hline\\
\endhead

% Tabellenkopf erste Seite
\hline
\textbf{\position} & \textbf{\artikelnummer} & \textbf{\bezeichnung} & \textbf{\menge} & \textbf{\einzelpreis} & \textbf{\gesamtpreis} \\
\hline\\[-0.5em]
\endfirsthead

% Tabellenende
\\
\multicolumn{6}{@{}r@{}}{\weiteraufnaechsterseite}
\endfoot

% Tabellenende letzte Seite
\hline\\
\multicolumn{5}{@{}l}{\nettobetrag} & <%subtotal%> \currency\\
<%foreach tax%>
\multicolumn{5}{@{}l}{<%taxdescription%>} & <%tax%> \currency\\
<%end tax%>
\multicolumn{5}{@{}l}{\textbf{\schlussbetrag}} &  \textbf{<%ordtotal%>} \currency\\
\hline\hline\\
\endlastfoot

% eigentliche Tabelle
<%foreach number%>
  <%runningnumber%> &
  <%number%> &
  \textbf{<%description%>} &
  \raggedleft <%qty%> <%unit%> &
  <%sellprice%> \currency &
  \ifthenelse{\equal{<%p_discount%>}{0}}{}{\sffamily\scriptsize{(-<%p_discount%> \%)}}
  \ourfont{<%linetotal%> \currency} \\*  % kein Umbruch nach der ersten Zeile, damit Beschreibung und Langtext nicht getrennt werden

  <%if longdescription%> && \scriptsize <%longdescription%>\\<%end longdescription%>
  <%if reqdate%> && \scriptsize \lieferdatum: <%reqdate%>\\<%end reqdate%>
  <%if serialnumber%> && \scriptsize \seriennummer: <%serialnumber%>\\<%end serialnumber%>
  <%if ean%> && \scriptsize \ean: <%ean%>\\<%end ean%>
  <%if projectnumber%> && \scriptsize \projektnummer: <%projectnumber%>\\<%end projectnumber%>
  <%if customer_make%>
    <%foreach customer_make%>
      \ifthenelse{\equal{<%customer_make%>}{<%name%>}}{&& \kundenartnr: <%customer_model%>\\}{}
    <%end foreach%>
  <%end if%>
  \\[-0.8em]
<%end number%>

\end{longtable}


\vspace{0.2cm}

<%if notes%>
        \vspace{5mm}
        <%notes%>
        \vspace{5mm}
<%end if%>


\textit{\auftragpruefen} \\ \\

\gruesse \\ \\ \\
  <%employee_name%>

\end{document}

